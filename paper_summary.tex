\documentclass[12pt]{article}
\usepackage[margin=1in]{geometry}
\usepackage{amsmath,amssymb}
\usepackage{lmodern}
\usepackage[T1]{fontenc}
% \usepackage{parskip}
\usepackage{natbib}
\usepackage{setspace}

\renewcommand{\hat}{\widehat}
\renewcommand{\tilde}{\widetilde} 

\title{Nickell Bias in Panel Local Projection: \\Financial Crises Are Worse Than You Think}
\author{Ziwei Mei, Liugang Sheng, Zhentao Shi}
\date{}


\begin{document}

\maketitle

\onehalfspacing

\section{Motivation}

In recent years, Local projection (LP) \citep{jorda2005estimation} has been one of the most important estimation techniques in macroeconomics. 
Panel LP (pLP) with country fixed effects (FE) is a standard tool for measuring how financial crises affect output. The approach regresses $y_{i,t+h}$ on crisis indicators $x_{i,t}$ and absorbs unit heterogeneity with FE. The paper shows that, even when the lagged output $y_{i,t}$ is not included in the regression, the dynamic structure inherent in predictive specifications makes the FE estimator by within-group transformations suffer from the Nickell bias. Therefore, the nominal significance level of the $t$-tests from the FE estimator no longer matches the empirical rejection probabilities. In practice, FE understates crisis-driven output losses.


\section{Intrinsic Nickell Bias}

To demonstrate the presence of the Nickell bias, we study a panel VAR(1) prototype:
\begin{align*}
 y_{i,t+1} &= \mu_i^{(0)y} + \beta^{(0)} x_{i,t+1} + u_{i,t+1}^y, \\
 x_{i,t+1} &= \mu_i^x + \rho x_{i,t} + u_{i,t+1}^x .
\end{align*}
Iterating the system implies that $y_{i,t+h}$ depends on current and lagged innovations in $x_{i,t}$, so the composite error $e_{i,t+h}^{(h)}$ in the pLP regression
$$
y_{i,t+h} = \mu_i^{(h)y} + \beta^{(h)} x_{i,t} + u_{i,t+h}^y, \quad h \in \{1,2 \ldots, H\} 
$$
violates \emph{strict exogeneity}, since $\mathbb{E}[e_{i,t+h}^{(h)} \mid x_{i,1}, x_{i,2}, ..., x_{i,T}] \neq 0$ whenever $\beta^{(0)} \neq 0$. Therefore, the FE estimator is biased.

Proposition~1 of the paper derives the analytic bias of the FE estimator (denoted by $\hat{\beta}^{(h) }$):
\[
\hat{\beta}^{(h) } \stackrel{a}{\sim} \beta^{(h)} - \frac{\mathrm{bias}}{T} + \frac{\mathrm{Normal(0)}}{\sqrt{N T}},
\]
where ``$\mathrm{bias}$'' is a non-zero constant, and ``$\mathrm{Normal(0)}$'' is a normally distributed random variable with finite variance and centered at 0.
The bias is of order $1/T$, and the asymptotically normal term has an order $1/\sqrt{NT}$. In the joint limit with $N/T\to c \in (0,\infty)$, the bias and the asymptotically normal term share the same order, and thus the bias is not negligible in the asymptotic normal distribution. Positive persistence ($\rho>0$) yields \emph{attenuation} ($\hat{\beta}^{(h) }$ is biased toward zero), and the distortion enlarges with a larger horizon $h$ and with more persistent crisis indicators. The $t$-statistics for $\beta^{(h)}$ remain biased even with known asymptotic variances.

\section{Split-Panel Jackknife}
To remove the leading bias, we recommend the split-panel jackknife (SPJ):
\begin{equation}\label{eq:spj}
\tilde{\beta}^{(h)} = 2 \hat{\beta}^{(h) } - \frac{\hat{\beta}^{(h) }_a + \hat{\beta}^{(h) }_b}{2},    
\end{equation}
where the subscripts $a$ and $b$ denote usual FE estimates from the first and second halves (over the time dimension) of the panel. 


SPJ accommodates more general regressions with multiple control variables:
$$
y_{i,t+h} = \mu_i^{(h)y} + \beta^{(h)\prime} \mathbf{x}_{i,t} + u_{i,t+h}^y, \quad h \in \{1,2 \ldots, H\} 
$$
where $\mathbf{x}_{i,t}$ is a $K$-dimensional regressor, and thus $\beta^{(h)}$ is a $K$-dimensional parameter. The procedure remains the same, with the three FE estimates in \eqref{eq:spj}. Under standard LLN and CLT conditions with $N/T^3 \to 0$, our paper's Theorem~1 shows that
\[
\sqrt{N T}(\tilde{\beta}^{(h)} - \beta^{(h)}) \Rightarrow \mathcal{N}(0, \mathbf{\Sigma}),
\]
where $\mathbf{\Sigma}$ is a $K\times K$ matrix of the asymptotic variance. 
SPJ is asymptotically unbiased, thereby delivering valid $t$-test and Wald test. The estimator preserves the appeal of LPs---researchers are still able to specify the impulse response function with a focus only on the predictive regressions of $y_{i,t+h}$ on $\mathbf{x}_{i,t}$.

Practically, we recommend SPJ for panels with $T \ge 30$ and $N/T \le 10$.

\section{Empirical Implications}
Revisiting four influential cross-country crisis studies, the SPJ-based impulse responses exhibit larger and more persistent output losses than those implied by fixed effects. Hence, the conventional FE local projection systematically understates how severe and long-lasting financial crises are.

\section{Takeaway}

In pLP regressions, we argue against the widely used FE estimator due to the intrinsic Nickell bias in the dynamic setting. 
We recommend the SPJ estimator. It is easy to implement, and maintains the validity of the usual $t$-statistic-based hypothesis testing procedure.


We provide an R package for users to carry out the SPJ estimation with a one-line command. 

\singlespacing 
\bibliographystyle{chicagoa}
\small

\bibliography{FELP}


\end{document}
